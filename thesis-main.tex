\documentclass[12pt,a4paper,openright,twoside]{book}
\usepackage[utf8]{inputenc}
\usepackage{disi-thesis}
\usepackage{code-lstlistings}
\usepackage{notes}
\usepackage{shortcuts}
\usepackage{acronym}
\usepackage{import}
\school{\unibo}
\programme{Corso di Laurea Magistrale in Ingegneria e Scienze Informatiche}
\title{Fancy Title}
\author{Giacomo Accursi}
\date{\today}
\subject{Laboratorio di Sistemi Software}
\supervisor{Prof. Danilo Pianini}
\cosupervisor{Dott. Gianluca Aguzzi}
\session{IV}
\academicyear{2022-2023}

% Definition of acronyms
% \acrodef{IoT}{Internet of Thing}
% \acrodef{vm}[VM]{Virtual Machine}


\mainlinespacing{1.241}

\begin{document}

\frontmatter\frontispiece

\begin{abstract}	
Max 2000 characters, strict.
\end{abstract}

\begin{dedication} % this is optional
Optional. Max a few lines.
\end{dedication}

\begin{acknowledgements} % this is optional
Optional. Max 1 page.
\end{acknowledgements}

%----------------------------------------------------------------------------------------
\tableofcontents   
\listoffigures     % (optional) comment if empty
\lstlistoflistings % (optional) comment if empty
%----------------------------------------------------------------------------------------

\mainmatter

%----------------------------------------------------------------------------------------
\chapter{Introduction}
\label{chap:introduction}
%----------------------------------------------------------------------------------------

% Write your intro here.
% \sidenote{Add sidenotes in this way. They are named after the author of the thesis}

% You can use acronyms that your defined previously,
% such as \ac{IoT}.
% %
% If you use acronyms twice,
% they will be written in full only once
% (indeed, you can mention the \ac{IoT} now without it being fully explained).
% %
% In some cases, you may need a plural form of the acronym.
% %
% For instance,
% that you are discussing \acp{vm},
% you may need both \ac{vm} and \acp{vm}.

% \paragraph{Structure of the Thesis}

% \note{At the end, describe the structure of the paper}

% \chapter{State of the art}

% I suggest referencing stuff as follows: \cref{fig:random-image} or \Cref{fig:random-image}

% \begin{figure}
%     \centering
%     \includegraphics[width=.8\linewidth]{}
%     \caption{Some random image}
%     \label{fig:random-image}
% \end{figure}

% \section{Some cool topic}

% \chapter{Contribution}

% You may also put some code snippet (which is NOT float by default), eg: \cref{lst:random-code}.

% \lstinputlisting[float,language=Java,label={lst:random-code}]{listings/HelloWorld.java}


%----------------------------------------------------------------------------------------
% Discrete Event Simulation
%----------------------------------------------------------------------------------------
\chapter{Simulazioni ad Eventi Discreti}
\import{chapters/}{DES.tex}

%----------------------------------------------------------------------------------------
% BIBLIOGRAPHY
%----------------------------------------------------------------------------------------

\backmatter

%\nocite{*} % comment this to only show the referenced entries from the .bib file

\bibliographystyle{alpha}
\bibliography{bibliography}

\end{document}